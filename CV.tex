%%%%%%%%%%%%%%%%%%%%%%%%%%%%%%%%%%%%%%%%%%%%%%%%%%%%%%%%%%%%%%%%%%%%%%%%
%%%%%%%%%%%%%%%%%%%%%% Simple LaTeX CV Template %%%%%%%%%%%%%%%%%%%%%%%%
%%%%%%%%%%%%%%%%%%%%%%%%%%%%%%%%%%%%%%%%%%%%%%%%%%%%%%%%%%%%%%%%%%%%%%%%

%%%%%%%%%%%%%%%%%%%%%%%%%%%%%%%%%%%%%%%%%%%%%%%%%%%%%%%%%%%%%%%%%%%%%%%%
%% NOTE: If you find that it says                                     %%
%%                                                                    %%
%%                           1 of ??                                  %%
%%                                                                    %%
%% at the bottom of your first page, this means that the AUX file     %%
%% was not available when you ran LaTeX on this source. Simply RERUN  %%
%% LaTeX to get the ``??'' replaced with the number of the last page  %%
%% of the document. The AUX file will be generated on the first run   %%
%% of LaTeX and used on the second run to fill in all of the          %%
%% references.                                                        %%
%%%%%%%%%%%%%%%%%%%%%%%%%%%%%%%%%%%%%%%%%%%%%%%%%%%%%%%%%%%%%%%%%%%%%%%%

%%%%%%%%%%%%%%%%%%%%%%%%%%%% Document Setup %%%%%%%%%%%%%%%%%%%%%%%%%%%%

% Don't like 10pt? Try 11pt or 12pt
\documentclass[10pt]{article}

% This is a helpful package that puts math inside length specifications
\usepackage{calc}
\usepackage{pifont}
\usepackage{marvosym}


% Simpler bibsection for CV sections
% (thanks to natbib for inspiration)
\makeatletter
\newlength{\bibhang}
\setlength{\bibhang}{1em}
\newlength{\bibsep}
 {\@listi \global\bibsep\itemsep \global\advance\bibsep by\parsep}
\newenvironment{bibsection}%
        {\vspace{-\baselineskip}\begin{list}{}{%
       \setlength{\leftmargin}{\bibhang}%
       \setlength{\itemindent}{-\leftmargin}%
       \setlength{\itemsep}{\bibsep}%
       \setlength{\parsep}{\z@}%
        \setlength{\partopsep}{0pt}%
        \setlength{\topsep}{0pt}}}
        {\end{list}\vspace{-.6\baselineskip}}
\makeatother

% Layout: Puts the section titles on left side of page
\reversemarginpar

%
%         PAPER SIZE, PAGE NUMBER, AND DOCUMENT LAYOUT NOTES:
%
% The next \usepackage line changes the layout for CV style section
% headings as marginal notes. It also sets up the paper size as either
% letter or A4. By default, letter was used. If A4 paper is desired,
% comment out the letterpaper lines and uncomment the a4paper lines.
%
% As you can see, the margin widths and section title widths can be
% easily adjusted.
%
% ALSO: Notice that the includefoot option can be commented OUT in order
% to put the PAGE NUMBER *IN* the bottom margin. This will make the
% effective text area larger.
%
% IF YOU WISH TO REMOVE THE ``of LASTPAGE'' next to each page number,
% see the note about the +LP and -LP lines below. Comment out the +LP
% and uncomment the -LP.
%
% IF YOU WISH TO REMOVE PAGE NUMBERS, be sure that the includefoot line
% is uncommented and ALSO uncomment the \pagestyle{empty} a few lines
% below.
%

%% Use these lines for letter-sized paper
%\usepackage[paper=letterpaper,
%           %includefoot, % Uncomment to put page number above margin
%            marginparwidth=0.7in,     % Length of section titles
%            marginparsep=.05in,       % Space between titles and text
%            margin=0.5in,               % 1 inch margins
%            includemp]{geometry}

% Use these lines for A4-sized paper
\usepackage[paper=a4paper,
            %includefoot, % Uncomment to put page number above margin
            marginparwidth=24mm,    % Length of section titles
            marginparsep=1mm,       % Space between titles and text
            margin=15mm,              % 25mm margins
            includemp]{geometry}

%% More layout: Get rid of indenting throughout entire document
\setlength{\parindent}{0in}

%% This gives us fun enumeration environments. compactitem will be nice.
\usepackage{paralist}
\usepackage[shortlabels]{enumitem}
% \usepackage[misc]{ifsym}
%% Reference the last page in the page number
%
% NOTE: comment the +LP line and uncomment the -LP line to have page
%       numbers without the ``of ##'' last page reference)
%
% NOTE: uncomment the \pagestyle{empty} line to get rid of all page
%       numbers (make sure includefoot is commented out above)
%
\usepackage{fancyhdr,lastpage}
\pagestyle{fancy}
%\pagestyle{empty}      % Uncomment this to get rid of page numbers
\fancyhf{}\renewcommand{\headrulewidth}{0pt}
\fancyfootoffset{\marginparsep+\marginparwidth}
\newlength{\footpageshift}
\setlength{\footpageshift}
          {0.1\textwidth+0.1\marginparsep+0.1\marginparwidth-2in}
\lfoot{\hspace{\footpageshift}%
       \parbox{3.5in}{\, \hfill %
                    \arabic{page} of \protect\pageref*{LastPage} % +LP
%                    \arabic{page}                               % -LP
                    \hfill \,}}

% Finally, give us PDF bookmarks
\usepackage{color,hyperref}
\definecolor{darkblue}{rgb}{0.0,0.0,0.3}
\hypersetup{colorlinks,breaklinks,
            linkcolor=darkblue,urlcolor=darkblue,
            anchorcolor=darkblue,citecolor=darkblue}

%%%%%%%%%%%%%%%%%%%%%%%% End Document Setup %%%%%%%%%%%%%%%%%%%%%%%%%%%%


%%%%%%%%%%%%%%%%%%%%%%%%%%% Helper Commands %%%%%%%%%%%%%%%%%%%%%%%%%%%%

% The title (name) with a horizontal rule under it
%
% Usage: \makeheading{name}
%
% Place at top of document. It should be the first thing.
\newcommand{\makeheading}[1]%
        {\hspace*{-\marginparsep minus \marginparwidth}%
         \begin{minipage}[t]{\textwidth+\marginparwidth+\marginparsep}%
                {\large \bfseries #1}\\[-0.15\baselineskip]%
                 \rule{\columnwidth}{1pt}%
         \end{minipage}}

% The section headings
%
% Usage: \section{section name}
%
% Follow this section IMMEDIATELY with the first line of the section
% text. Do not put whitespace in between. That is, do this:
%
%       \section{My Information}
%       Here is my information.
%
% and NOT this:
%
%       \section{My Information}
%
%       Here is my information.
%
% Otherwise the top of the section header will not line up with the top
% of the section. Of course, using a single comment character (%) on
% empty lines allows for the function of the first example with the
% readability of the second example.
\renewcommand{\section}[2]%
        {\pagebreak[1]\vspace{1.5\baselineskip}%
         \phantomsection\addcontentsline{toc}{section}{#1}%
         \hspace{0in}%
         \marginpar{
         \raggedright \scshape #1}#2}

% An itemize-style list with lots of space between items
\newenvironment{outerlist}[1][\enskip\textbullet]%
        {\begin{itemize}[#1]}{\end{itemize}%
         \vspace{-0.6\baselineskip}}

% An environment IDENTICAL to outerlist that has better pre-list spacing
% when used as the first thing in a \section
\newenvironment{lonelist}[1][\enskip\textbullet]%
        {\vspace{-\baselineskip}\begin{list}{#1}{%
        \setlength{\partopsep}{0pt}%
        \setlength{\topsep}{0pt}}}
        {\end{list}\vspace{-.6\baselineskip}}

% An itemize-style list with little space between items
% \newenvironment{innerlist}[1][\enskip\textbullet]%
\newenvironment{innerlist}[1][\enskip$\circ$]%
        {\begin{compactitem}[#1]}{\end{compactitem}}

% An environment IDENTICAL to innerlist that has better pre-list spacing
% when used as the first thing in a \section
\newenvironment{loneinnerlist}[1][\enskip\textbullet]%
        {\vspace{-\baselineskip}\begin{compactitem}[#1]}
        {\end{compactitem}\vspace{-.6\baselineskip}}

% To add some paragraph space between lines.
% This also tells LaTeX to preferably break a page on one of these gaps
% if there is a needed pagebreak nearby.
\newcommand{\blankline}{\quad\pagebreak[2]}

% Uses hyperref to link DOI
\newcommand\doilink[1]{\href{http://dx.doi.org/#1}{#1}}
\newcommand\doi[1]{doi:\doilink{#1}}


%%%%%%%%%%%%%%%%%%%%%%%% End Helper Commands %%%%%%%%%%%%%%%%%%%%%%%%%%%

%%%%%%%%%%%%%%%%%%%%%%%%% Begin CV Document %%%%%%%%%%%%%%%%%%%%%%%%%%%%

%\hyphenpenalty = 9999
\def\vs{\vspace{-0.1in}}
\begin{document}
% \makeheading{Curriculum Vitae\\ [0.3cm] TIEP HUU VU\quad~~~~~~\quad\quad\quad\quad\quad\quad\quad\quad\quad\quad\quad\quad\quad\quad{\small Last update: December 17, 2015}}
\makeheading{Jerónimo Valencia Porras }


\section{Contact Information}
\newlength{\rcollength}\setlength{\rcollength}{3 in}
\vs

\texttt{Linkedin:}\href{https://www.linkedin.com/in/jeronimo-valencia-porras/}{www.linkedin.com/in/jeronimo-valencia-porras/}\\
\texttt{GitHub:}\href{http://www.github.com/jeronimovalencia}{www.github.com/jeronimovalencia}\\
{\large\Letter} \texttt{Personal E-mail:}\href{mailto:jeronimovalencia9711gmail.com}{jeronimovalencia9711@gmail.com}\\
{\large\Letter}
\texttt{Academic E-mail:}\href{mailto:jeronimovalencia9711gmail.com}{j2valenc@uwaterloo.ca}\\


%\section{Research Background} % (fold)
%\label{sec:research_backg}
%\vspace{-0.25in}
%\begin{outerlist}
%  \item {\bf ??}: more descriptions here.
%  \item {\bf ??}:
%\end{outerlist}
% section research_backg (end)
%% =========  ==============================
\section{Education}
    {\textbf{Ph.D., Combinatorics and Optimization, University of Waterloo}} \hfill Expected Aug. 2026.
    
    \vspace{0.5cm}
    \href{https://matematicas.uniandes.edu.co/}{\textbf{M.Sc, Mathematics, Universidad de los Andes}}, Bogotá, Colombia \hfill Apr. 2022
    \begin{outerlist}
        \item {\it Relevant courses:} Polytope combinatorics, Tropical geometry, \\ Toric varieties. 
        \item {\it Thesis:} {Ehrhart theory of Lattice Path Matroid polytopes.}   
        \item { \it Advisor:} Prof. Carolina Benedetti Velásquez, Universidad de los Andes. \\ { \it Co-advisor:} Prof. Kolja Knauer, Universitat de Barcelona.
    \end{outerlist}
    
    \vspace{0.5cm}
     \href{https://matematicas.uniandes.edu.co/}{\textbf{B.Sc, Mathematics, Universidad de los Andes}}, Bogotá, Colombia \hfill Apr. 2020
    \begin{outerlist} 
        \item {\it Relevant courses:} Algebraic combinatorics, Commutative algebra,\\ Algebraic geometry.
        \item {\it Thesis:} {Álgebras de Hopf de cocientes de matroides}
        \item {\it Advisor:} Prof. Carolina Benedetti Velásquez, Universidad de los Andes.
         
    \end{outerlist}
    
    \vspace{0.5cm}
    
    \href{https://fisica.uniandes.edu.co/}{\textbf{B.Sc, Physics, Universidad de los Andes}}, Bogotá, Colombia \hfill Apr. 2019 
    \begin{outerlist}
        \item {\it Relevant courses:} Statistical mechanics, Topics in Statistical mechanics,\\ Group theory in Quantum mechanics.
        \item {\it Thesis:} {Emergent chaos in the verge of phase transitions.}
        \item {\it Advisor:} Prof. Gabriel Téllez Acosta, Universidad de los Andes.
    \end{outerlist}

    %% =========  ==============================
\section{In-course Projects} % (fold)
\vspace{-0.25in}
\begin{outerlist}
    \item {\bf Matroids in Tropical Geometry} \hfill Dec. 2020
    \begin{innerlist}
        \item {\it Course: } Tropical Geometry.
        \item {\it Advisor: } Prof. Johannes Rau.
    \end{innerlist}
    \item {\bf Volume of simplicial complex polytopes}, with Eliana Tolosa. \hfill Dec. 2020
    \begin{innerlist}
        \item {\it Course: } Polytope combinatorics.
        \item {\it Advisors: } Prof. Carolina Benedetti Velásquez and Prof. Federico Ardila.
    \end{innerlist}
    \item {\bf Knot invariants and graphical matroids}, with Felipe Rueda. \hfill Jun. 2019
    \begin{innerlist}
        \item {\it Course: } Algebraic combinatorics.
        \item {\it Advisor: } Prof. Carolina Benedetti Velásquez.
    \end{innerlist}
\end{outerlist}
%% =========  ==============================
\section{Research Experience} % (fold)
\label{sec:research_exper}
\vspace{-0.25in}
\begin{outerlist}
    \item {\bf Real structures on Lattice Path Matroids} \hfill Jun. 2021 - Present\\
    Universidad de los Andes. 
    \begin{innerlist}
        \item {Explicit calculations of real structures on uniform matroids.} 
        \item {\it Joint work with Prof. Johannes Rau.}
    \end{innerlist}

    \item {\bf $f^*$-vectors} \hfill Mar. 2021 - Present
    \begin{innerlist}
        \item Combinatorial characterization of $f^*$-vectors for lattice polytopes. 
        \item {\it Joint work with Prof. Matthias Beck, Danai Deligeorgaki and \\ Max Hlavacek.}
    \end{innerlist}
    
    \item {\bf Random generation of probability problems } \hfill Jun. - Aug. 2020\\
    Universidad de los Andes. 
    \begin{innerlist}
      \item {Implementation of a computer program to generate \\ probability exercises with random questions and answers.}
      \item \textit{Work under the supervision of Prof. Alexander Getmanenko.}
    \end{innerlist}
\end{outerlist}
\newpage
%% =========  ==============================
\section{Conferences and Seminars} % (fold)
\vspace{-0.25in}
\begin{outerlist}
    \item {\bf Encuentro Colombiano de Combinatoria 2022}\hfill Jun. 2022 \\
    \textit{Contributed talk:} On Ehrhart theory of Lattice Path Matroid polytopes\\
    \textit{Teaching Assistant:} Minicourse: Real tropical geometry
    \item {\bf Seminar: Symplectic Matroids} \hfill Aug. - Dec. 2021\\
    \textit{Co-organizer, with Prof. Carolina Benedetti Velásquez}\\
    Department of Mathematics, Universidad de los Andes. 
        \item {\bf Fragments of algebraic graph theory} \hfill Jan. - Mar. 2021 \\
        Barcelona Graduate School of Mathematics. 
    \item {\bf Research Encounters in Algebraic and Combinatorial Topics} \hfill Feb. - Mar. 2021 \\ Online conference: \href{https://sites.google.com/view/react-2021}{https://sites.google.com/view/react-2021}
    \item {\bf Seminar: Polymatroids} \hfill Aug. - Dec. 2020\\
    \textit{Co-organizer, with Prof. Carolina Benedetti Velásquez}\\
    Department of Mathematics, Universidad de los Andes.
    \item {\bf Quantum Symmetries} \hfill Jun. - Jul. 2019 \\ Universidad de los Andes, CIMPA Bogotá 2019. 
    \item {\bf Seminar: Statistical Mechanics} \hfill Aug. - Dec. 2018\\
    \textit{Organizer:  Prof. Gabriel Téllez Acosta}\\
    Department of Physics, Universidad de los Andes.
\end{outerlist}
%% ================== block:  ==========================
\section{Awards}
\vspace{-0.25in}
\begin{outerlist}
    \item {\bf Full tuiton - Masters.} \hfill Jan. 2020 - Dec. 2021\\
    Department of Mathematics, Universidad de los Andes. 
\end{outerlist}

%% ================== block:  ==========================
\section{Teaching}
\vspace{-.25in}
\begin{outerlist}
    \item {\bf Teaching Assistant} \hfill Oct. - Nov. 2021\\ Mathematics Sin Fronteras, Brown University.
    \begin{innerlist}
        \item {\it Course:} Lattice paths, linear algebra and combinatorics. \\ In charge of Prof. Carolina Benedetti Velásquez.
        \item[-] Translate course materials. 
        \item[-] Solve student questions on weekly homework exercises.
    \end{innerlist}
    \item{\bf Teaching Assistant} \hfill 2017-2021\\ Department of Mathematics, Universidad de los Andes.
    \begin{innerlist}
        \item {\it Courses:} Differential calculus, Integral calculus, Multivariate \\ calculus (online), Linear Algebra, Commutative Algebra.
        \item[-] Plan and moderate weekly sessions of exercises.
        \item[-] Provide office hours to solve course questions. 
        \item[-] Design weekly quices and assignments, and deliver feedback to \\ the students. 
        \item[-] Experience teaching online. 
    \end{innerlist}
    \item {\bf Tutor at \textit{Pentágono}} \hfill 2017-2021\\ Department of Mathematics, Universidad de los Andes.
    \begin{innerlist}
        \item[-] Provide office hours to help students with their undergraduate \\ mathematics problems.
    \end{innerlist}
    \item {\bf Tutor at \textit{Clínica de Problemas}} \hfill 2017-2019\\ Department of Physics, Universidad de los Andes.
    \begin{innerlist}
        \item[-] Provide office hours to help students with their undergraduate \\ physics problems.
    \end{innerlist}
    
\end{outerlist}
%% =========  ==============================
\section{Publications}
\vspace{-0.25in}
\begin{outerlist}
\item {\bf Sequence of volumes of Schubert matroids of rank 2}. \hfill Sept. 2021\\Online Encyclopedia of Integer Sequences: \href{https://oeis.org/search?q=A347976}{A347976}
\end{outerlist}
%% =========  ==============================
\section{Preprints}
\vspace{-0.25in}
\begin{outerlist}
\item {\bf Ehrhart theory of lattice path matroid polytopes } \\ \textit{with Carolina Benedetti Velásquez and Kolja Knauer}. In progress.
\item {\bf Inequalities for $f^*$-vectors of lattice polytopes } \\ \textit{with Matthias Beck, Danai Deligeorgaki and Max Hlavacek}. In progress.
\end{outerlist}

%% ==============================================================
\section{Talks}
\vspace{-0.25in}
\begin{outerlist}
    \item {\bf Ehrhart theory of Lattice Path Matroid polytopes } \hfill Nov. 2021\\
    Seminario Sabanero de Combinatoria, Bogotá, Colombia. 
    \item {\bf Matroids in Physics } \hfill May. 2021\\
    Quantum Mechanics and Information Theory seminar. \\ Department of Physics, Universidad de los Andes. 
    \item {\bf Inversion of formal power series and combinatorics} \hfill Apr. 2020\\
    Student's seminar.\\ Department of Mathematics, Universidad de los Andes.
\end{outerlist}

%% =========  ==============================
\section{Languages}
\vspace{-0.25in}
\begin{outerlist}
\item {\bf Spanish:} Native.
\item {\bf English:} Advanced (TOEFL iBT score: 109/120).
\end{outerlist}

%% =========  ==============================
\section{Technical Skills} % (fold)
\vspace{-0.25in}
\begin{outerlist}
  \item {\it Programming Languages}: Python, Sage, \LaTeX, R, C++.
  \item {\it Online teaching:} Zoom, Microsoft Teams. 
  \item {\it Technical Softwares}: Mathematica, VS Code.
\end{outerlist}

%% =========  ==============================
\section{Personal Interests}
\vspace{-0.25in}
\begin{outerlist}
    \item {\it Sports:} 3-cushion billiards, rock climbing, volleyball.
    \item {\it Hobbies:} Rubik's cube and its variants, board games, retro videogames,\\ indoor plant care, cooking. 
    \item {\it Books:} mystery and suspense novels, classic novels, science fiction. 
\end{outerlist}


%% =========  ==============================
\section{References}
\vspace{-0.25in}
\begin{outerlist}
\item Carolina Benedetti Valásquez, Assistant Professor, Universidad de los Andes. \\ {\large\Letter} c.benedetti@uniandes.edu.co
\item Kolja Knauer, Associate Professor, Universitat de Barcelona\\ {\large\Letter} kolja.knauer@googlemail.com
\item Johannes Rau, Assistant Professor, Universidad de los Andes. \\ {\large\Letter} j.rau@uniandes.edu.co
\end{outerlist}


%% =========  ==============================
\end{document}
%% ==============================================================
